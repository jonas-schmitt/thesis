\section{Formal Languages and Grammars}
In the previous section, we have introduced the theoretical background necessary to understand the fundamentals of multigrid methods.
The goal of this thesis is to develop a formal language that can express multigrid solvers of variable structure.
In general the \emph{alphabet} of a formal language is a non-empty set of symbols $\Sigma$.
Based on $\Sigma$ \emph{strings} can be created, which represent finite sequences of symbols.
For instance the alphabet $\Sigma = \left\{a, b\right\}$ contains only the symbols $a$ and $b$, as such $aabbb$ and $abaab$ are both strings on the alphabet $\Sigma$.
The length of a string, denoted by $|\cdot|$, is equal to the number of symbols contained in it. 
A concatenation of two strings 
\begin{equation}
	\begin{split}
		v & = a_1 a_2 \cdots a_n \\
		w & = b_1 b_2 \cdots b_m
	\end{split}
\end{equation}
is performed by appending the second string to the right of the first string such that
\begin{equation}
	vw = a_1 a_2 \cdots a_n b_1 b_2 \cdots b_m,
\end{equation}
The length of the concatenated string is then obviously equal to the sum of the length of the individual string, which means
\begin{equation}
|vw| = |v| + |w|.
\end{equation}
Finally, we define the empty string $\lambda$ as
\begin{equation}
	\begin{split}
		& |\lambda| = 0 \\
		& v \lambda = \lambda v = v,
	\end{split}
\end{equation}
where $v$ is a string on an arbitrary alphabet.
Assume $\Sigma$ is an alphabet, then $\Sigma^*$ is the set of strings obtained by concatenating zero or an arbitrary number of symbols from $\Sigma$.
Consequently, $\Sigma^*$ always contains the empty string $\lambda$.
To exclude $\lambda$ from the set, we define $\Sigma^+ = \Sigma^* \setminus \lambda$.
Since the number of strings that can be created from an alphabet through concatenation is infinite, both $\Sigma^+$ and $\Sigma^*$ represent infinite sets.
For example again let $\Sigma = \left\{a, b\right\}$ then
\begin{equation*}
	\Sigma^{*} = \left\{\lambda, a, b, aa, ab, ba, bb, aaa, aab, aba, \dots \right\}
\end{equation*} 
In general, for a given alphabet the \emph{language} $L$ is defined as a subset of $\Sigma^*$ (or $\Sigma^+$)
\begin{equation}
	L \supset \Sigma^*.
	\label{eq:language-basic-definition}
\end{equation}
However, in practice we usually want to obtain a language $L_G$ that represents a specific subset of $\Sigma^*$.
One way to achieve this is to define a set of rules that generate $L_G$.
These rules can be formally specified in form of a \emph{grammar}.
\begin{definition}[Grammar]
	\begin{equation*}
		G = \left(V, T, S, P \right),
	\end{equation*}
	where $V$ is a finite set of \emph{variables},
	$T$ a finite set of \emph{terminal symbols},
	$S \in V$ is the \emph{start} variable and 
	$P$ a finite set of \emph{productions} or \emph{production rules}.
	We also assume that $V \neq \emptyset$, $T \neq \emptyset$ and $V \cap T = \emptyset$.
\end{definition}
The core of a grammar are the productions $P$, which are usually specified as a set of mappings of the form
\begin{equation*}
	x \to y,
\end{equation*}
where $x \in \left(V \cup T\right)^+$ and $y \in \left(V \cup T\right)^*$.
Each production rule $x \to y$ can then be applied in the following way.
Given a string $u$ of the form 
\begin{equation}
	u = vxw,
\end{equation}
we can replace $x$ with $y$ to \emph{derive} a new string
\begin{equation}
	u' = vyw.
\end{equation}
We usually write this transformation as $u \Rightarrow u'$.

