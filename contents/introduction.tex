%TODO The following is about general algorithm design
Many of the most fundamental laws of nature can be formulated as partial differential equations (PDEs).
Understanding these equations is, therefore, foundational for many branches of science and engineering.
Since, for many PDEs, it is unknown whether a general solution exists, the efficient approximate solution of these equations represents one of humanity's greatest challenges.
Since the invention of modern computers, great efforts have been made to develop PDE solvers with ever-increasing efficiency.
As a result of this effort, computer simulations nowadays represent an essential tool for both researchers and engineers.
However, leveraging the power of simulation-based methods, in many cases, requires designing solvers that achieve the highest possible degree of efficiency.
Unfortunately, this task often not only requires a great deal of expertise and domain knowledge that only a limited group of people possesses, but often also uncountable working hours need to be invested for its accomplishment.
All this makes the manual design and implementation of efficient PDE solvers a challenging and costly endeavor.
The automation of manual labor has always been one of the greatest incentives of the technical revolution since the first industrial revolution.
However, in contrast to previous technological advancements, which were mostly concerned with freeing people from physical labor, the development of computing devices with ever-increasing power and speed has led to a point where the automation even of challenging cognitive tasks has started to come into reach.
Within the last decade, artificial intelligence (AI) methods have demonstrated super-human performance in a number of different domains, such as game playing and natural language processing.
While artificial intelligence (AI) methods have demonstrated super-human performance in numerous applications, such as image processing, game playing, and natural language processing.
However, in contrast to the widespread success in these domains, AI-based techniques for automated algorithm design have only been leveraged in a limited number of applications such as the design of SAT-solvers~\cite{khudabukhsh2016satenstein}, metaheuristics~\cite{stuetzle2019automated}, and matrix-multiplication algorithms~\cite{fawzi2022discovering}.
%In general, designing an algorithm based on a set of global parameters can be summarized under the term \emph{top-down}, as the choice of each parameter has a global effect on the method's behavior and, thus, might affect multiple computational steps simultaneously. 
%For instance, choosing a different value for the parameter $\mu$ in Algorithm~\ref{alg:multigrid-cycle} influences the number of recursive descents on every level and, thus, leads to a globally different type of multigrid cycle.
%In contrast, a bottom-up approach is characterized by the ability to change each computational step within an algorithm without affecting the behavior of any other.