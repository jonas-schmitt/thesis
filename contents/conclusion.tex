\paragraph{Conclusion}
We have started this thesis with an introduction to automated algorithm design and its application to different domains.
Since our main goal has been to harness the potential of evolutionary program synthesis for the automated design of efficient multigrid methods, the question of whether we have achieved our original goals remains to be answered.
After establishing a theoretical foundation about multigrid methods, formal languages, and evolutionary program synthesis, in Chapter~\ref{chapter:multigrid-formal-language}, we have derived a novel context-free grammar, which enables us to construct sequences of multigrid operations that can not be obtained with the classical formulation of these methods.
Building on this foundation, we have presented a prototypical implementation of our evolutionary program synthesis tool, called \emph{EvoStencils}, which automates the design and implementation of efficient and generalizable multigrid-based PDE solvers by leveraging the capabilities of the evolutionary computation library DEAP~\cite{rainville2012deap} and the ExaStencils~\cite{lengauer2020exastencils} code generation framework.
In Chapter~\ref{chapter:experiments}, we could then finally demonstrate that this approach yields efficient and generalizable multigrid methods for different PDEs, whereby using the example of the indefinite Helmholtz equation, our automatically-designed solvers were able to achieve super-human performance in an outstandingly-difficult benchmark problem for the application of numerical PDE solvers\footnote{For this result, the corresponding paper~\cite{schmitt2022evolving} has been awarded the 2022 Humies Gold Award for Human-Competitive Results (\url{https://www.human-competitive.org/awards})}.
While we believe that this lays the foundation for the utilization of automated algorithm design methods within the domain of numerical PDE solvers, the multigrid methods considered in this work represent only a tiny fraction of this vast research area.
%We, therefore, believe that there exists a multitude of promising extensions of this work, some of which we briefly want to discuss in the following.
A promising extension of this approach would thus be its application to other multigrid variants or even other classes of numerical solvers.
While this thesis focuses on classical geometric multigrid (GMG) methods, since the invention of these methods, several other variants tailored to different use cases have been developed.
One example is the full-approximation scheme (FAS), which can be considered a non-linear version of multigrid~\cite{trottenberg2000multigrid,briggs2000multigrid}.
While the formulation of an FAS method requires replacing both the smoother and coarse-grid solver with non-linear variants that are, for instance, based on Newton's or Picard's method, the application of our program synthesis approach requires only a minimal amount of adaption.
To illustrate this, Section~\ref{appendix:fas} shows the necessary adaptions of our original state transition functions and grammar productions for generating FAS-style multigrid methods. 
As it can be seen there, apart from adjusting the respective productions, only the two functions \textsc{coarsening} and \textsc{cgc} need to be changed, while, additionally, a single new state transition function \textsc{cgs} for the application of the coarse-grid solver needs to be provided. 
Note that since the application of the operator $A_h$ no longer represents a linear operation in the case of FAS, we instead denote it as a function application.
Another promising idea would be an extension of our approach to full-multigrid methods.
While in Section~\ref{sec:experiments-part1}, we have demonstrated that the multigrid methods designed with our approach can solve different PDEs faster than common cycles, in many linear cases, including Poisson's equation, FMG potentially represents the fastest and most efficient solver available~\cite{trottenberg2000multigrid}.
Since FMG is based on applying multigrid cycles on different levels of a given discretization hierarchy, one possibility would be to apply our evolutionary program synthesis method to each of these individual cycles.
While this approach would result in an even larger search space than the ones considered in this work, it could yield methods that achieve an even higher degree of efficiency in solving PDE-based problems.
Finally, another class of multigrid methods not yet considered in this work is algebraic multigrid (AMG).
In contrast to GMG, AMG methods derive their operations directly from an algebraic formulation of the system of linear equations in the form of a (usually sparse) matrix.
While this makes these methods fundamentally different from the multigrid methods considered in this work, the actual algorithmic structure of AMG methods is similar to its geometric counterpart.
We could, therefore, formulate a grammar structurally similar to the one shown in Algorithm~\ref{table:multigrid-grammar} that replaces each individual operation applied within their productions with their algebraic equivalent, which would allow us to utilize the same program synthesis approach for the automated design of AMG methods.
Apart from considering different multigrid variants, another interesting direction would be to incorporate additional components and even completely different classes of solvers into the algorithm design space.
As Figure~\ref{fig:overview-ai-based-methods} illustrates, recently, a whole new class of AI-based solvers has been developed, such as data-driven~\cite{thuerey2020deep} and physics-informed~\cite{karniadakis2021physics} surrogate models.
While these methods show promise in individual domains, it is not yet clear how these methods can be integrated or combined with established numerical solvers.
Automated algorithm design offers an attractive solution to this problem by integrating these methods into the corresponding search space, for instance, in the form of the class of multigrid grammars introduced in this thesis.
For instance, an idea proposed in~\cite{markidis2021old} is to utilize physics-informed neural networks (PINNs) as a coarse-grid solver within multigrid methods.
A second complementary extension would be to target the current limitations in the evaluation accuracy of predictive models, which have been briefly discussed in Section~\ref{sec:fitness-evaluation-and-selection}.
While local Fourier analysis (LFA) has been successfully applied to different applications~\cite{rodrigo2017validity}, the missing availability of broadly-tested open-source tools for its automated use currently limits its applicability to the approach presented in this thesis~\cite{schmitt2020constructing}.
In case this situation might not change in the future, an alternative would be to use a statistical model to predict the quality of a certain multigrid method based on a history of samples.
One such approach, which has proven to be successful in other domains of automated algorithm design, such as automated machine learning (AutoML)~\cite{kotthoff2019auto,snoek2012practical}, is Bayesian optimization (BO)~\cite{frazier2018tutorial}.
While, in contrast to evolutionary algorithms, there is usually a limit to the number of parameters that can be optimized using BO, recent work in the field of AutoML demonstrates that these methods can also be scaled to grammar-based search spaces~\cite{schrodi2022towards}.
Finally, if we think in broader terms and consider the complete field of scientific computing, automation does not need to stop at the solver level but could include the full simulation design space.
For instance, in many cases, the choice of a suitable solver is not the main issue, but the considered system consists of different physical domains that each need to be simulated separately~\cite{gomes2018co}.
Only the coupling of these submodules then yields an understanding of the dynamic behavior of the complete system.
Similar to the recent success of automated algorithm design methods in the domain of data science, where systems exhibit a similarly high degree of complexity, the development of a unified approach for the automated design of simulation-based systems could yield great benefits in the future.