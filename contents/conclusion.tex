\paragraph{Conclusion}
We have started this thesis with an introduction to automated algorithm design methods and their successful application to different domains.
Since our main goal was to harness the potential of these methods for the automated design of multigrid methods for solving partial differential equations (PDEs), the question of whether we have achieved our original goals remains to be answered.
After establishing a theoretical foundation on multigrid methods, formal languages, and evolutionary program synthesis, in Chapter~\ref{chapter:multigrid-formal-language}, we have derived a novel context-free grammar, which enables us to construct sequences of multigrid operations that could not yet be obtained with the classical formulation of these methods.
Building upon this foundation, we have then presented a prototypical implementation of our evolutionary program synthesis tool, called \emph{EvoStencils}, which automates the design and implementation of efficient and generalizable multigrid-based PDE solvers by leveraging the capabilities of the evolutionary computation library DEAP~\cite{rainville2012deap} and the ExaStencils~\cite{lengauer2020exastencils} code generation framework.
In Chapter~\ref{chapter:experiments}, we were then able to demonstrate that this approach yields efficient multigrid methods for different PDEs, whereby using the example of the indefinite Helmholtz equation, our automatically-designed solvers were even able to achieve super-human performance in an outstandingly-difficult benchmark problem for the application of numerical PDE solvers\footnote{For this result, the corresponding paper has been awarded the 2022 Humies Gold Award for Human-Competitive Results (\url{https://www.human-competitive.org/awards})}.
While we believe that this lays the foundation for the utilization of automated algorithm design methods within the domain of numerical PDE solvers, these multigrid methods only represent a tiny fraction of this vast research area.
%We, therefore, believe that there exists a multitude of promising extensions of this work, some of which we briefly want to discuss in the following.
Therefore, a promising extension of this work is the application of the presented approach to other multigrid variants or even other classes of numerical solvers.
While this thesis, as a starting point, focuses on classical geometric multigrid methods, since the invention of these methods, several other variants have been developed which are tailored to other use cases.
One example is the full-approximation scheme (FAS), which can be considered a non-linear version of multigrid~\cite{trottenberg2000multigrid,briggs2000multigrid}.
While the formulation of the FAS method requires replacing both the smoother and coarse-grid solver with non-linear variants that are often based on Newton's or Picard's method, the application of our evolutionary program synthesis requires only a minimal amount of adaption.
To illustrate this, Section~\ref{appendix:fas} shows the necessary adaptions of our original state transition functions and grammar productions for generating FAS-style multigrid methods. 
As it can be seen there, apart from updating the respective productions, essentially only the two functions \textsc{cycle} and \textsc{cgc} need to be changed, while we additionally provide a new state transition function \textsc{cgs} for the application of the coarse-grid solver. 
Note that since, in the case of FAS, the application of the operator $A_h$ no longer represents a linear operation, we instead denote it as a function application.
Another promising extension would be an extension of our approach to full-multigrid (FMG) methods.
While in Section~\ref{sec:experiments-part1}, we have demonstrated that multigrid cycles designed with our approach represent faster solvers than common multigrid cycles for different PDEs, in many cases, including Poisson's equation, FMG represents the fastest and most efficient available solver for linear problems~\cite{trottenberg2000multigrid}.
Since FMG is based on the application of multigrid cycles on different levels of a given discretization hierarchy, one possibility would be to apply our evolutionary program synthesis method to each of these individual cycles.
While this approach would lead to an even larger search space than the ones considered in this work, it could yield methods that achieve even higher efficiency in solving PDE-based problems.
Finally, another class of multigrid methods not yet considered in this work is algebraic multigrid (AMG).
In contrast to geometric multigrid, AMG methods derive all of their operations directly from an algebraic formulation of a system of linear equations in the form of a (usually sparse) matrix.
While this makes these methods fundamentally different from the multigrid methods considered in this work, the actual algorithmic structure of AMG methods is no different from their geometric counterparts.
We could, therefore, formulate a structurally-similar grammar to the one shown in Table~\ref{table:multigrid-grammar} that simply replaces each individual operation applied within their productions with their algebraic equivalent, which would allow us to apply the same approach for the automated design of AMG methods.
