Many of the most fundamental laws of nature can be formulated as partial differential equations (PDEs). 
Understanding these equations is, therefore, of exceptional importance for many branches of modern science and engineering. 
However, since the general solution of many PDEs is unknown, the efficient approximate solution of these equations is one of humanity's greatest challenges.
While multigrid represents one of the most effective methods for solving PDEs numerically, in many cases, the design of an efficient or at least working multigrid solver is an open problem.
This thesis demonstrates that grammar-guided genetic programming, an evolutionary program synthesis technique, can discover multigrid methods of unprecedented structure that achieve a high degree of efficiency and generalization.
For this purpose, we develop a novel context-free grammar that enables the automated generation of multigrid methods in a symbolically-manipulable formal language, based on which we can apply the same multigrid-based solver to problems of different sizes without having to adapt its internal structure.
Treating the automated design of an efficient multigrid method as a program synthesis task allows us to find novel sequences of multigrid operations, including the combination of different smoothing and coarse-grid correction steps on each level of the discretization hierarchy.
To prove the feasibility of this approach, we present its implementation in the form of the Python framework EvoStencils, which is freely available as open-source software.
This implementation comprises all steps from representing the algorithmic sequence of a multigrid method in the form of a directed acyclic graph of Python objects to its automatic generation and optimization using the capabilities of the code generation framework ExaStencils and the evolutionary computation library DEAP.
We furthermore describe how this implementation can be extended to yield multigrid methods that can efficiently solve multiple instances of the same PDE, thus achieving strong generalizability.
Even though generalization is one of the main goals of automated solver generation, artificial intelligence-based methods often fail to achieve it. 
While machine learning models have shown promise in replacing classical numerical solvers, they typically rely on a fixed-size neural network and can thus not easily be generalized to other problem sizes.
To speed up the evaluation of a large number of multigrid-based solvers, we derive a suitable distributed parallelization scheme based on the message-passing interface (MPI) that allows EvoStencils to leverage the computational power of modern clusters and supercomputers.
To investigate the effectiveness of our approach, we consider several different PDEs, including the indefinite Helmholtz equation, for which we obtain multigrid methods that achieve superior solving efficiency compared to classical multigrid cycles.
Moreover, some of the methods discovered with our approach are able to achieve convergence in the case of an extremely ill-conditioned Helmholtz problem, for which, at the same time, all known multigrid cycles fail to yield a converging solver.
Within our experiments, we also show that our implementation can be executed on recent clusters and supercomputers, such as SuperMUC-NG, currently one of Europe's largest supercomputing systems.
Finally, since our formal representation of multigrid methods can easily be translated into a human-readable format, we also perform an empirical analysis of the algorithmic features discovered with our evolutionary program synthesis approach.