Many of the most fundamental laws of nature can be formulated as partial differential equations (PDEs). Understanding these equations is, therefore, foundational for many branches of modern science and engineering. Since, for many PDEs, it is unknown whether a general solution exists, the efficient approximate solution of these equations represents one of humanity's greatest challenges. Within this thesis, we demonstrate that grammar-guided genetic programming (G3P) is capable of evolving multigrid methods of novel structure that achieve strong generalization and efficiency in solving PDE-based problems. 
Even though generalization is one of the main goals of automated solver generation, artificial intelligence-based methods often fail to achieve it. Recently, especially data-driven and physics-informed machine learning models have made enormous progress in solving PDEs. While these methods have shown promise in replacing classical numerical solvers, they typically rely on a fixed-size neural network and can, thus, not easily be generalized to other problem sizes.
In contrast, the representation of a solver in a symbolically-manipulable formal language allows us to apply the same method to problems of different size without adapting its internal structure. We utilize this property to implement an evolutionary search method that enables the systematic generalization of multigrid to a given problem domain. Furthermore, as our newly-developed formal representation of multigrid methods can easily be translated into a human-readable format, it can be understood and analyzed by a domain expert. To investigate whether our approach can be successfully applied to challenging PDEs, in this thesis, we consider an outstandingly-difficult benchmark problem, the indefinite Helmholtz equation, for which we evolve solvers that achieve superior generalization and efficiency compared to all known multigrid variants.
Finally, we demonstrate that our implementation of G3P, which is freely available in the form of the open-source library EvoStencils, can be scaled up on recent clusters and supercomputers. In particular in this thesis, we perform each experiment on eight nodes with a total number of 384 CPU cores of SuperMUC-NG, currently one of the largest supercomputing systems in Europe.