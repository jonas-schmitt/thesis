\section{Discretization of Partial Differential Equations}\label{sec:discretization}
Many problems in science and engineering can be modeled as partial differential equations (PDEs). %TODO insert references
A PDE is an equation that contains functions of one or multiple variables together with their partial derivatives.
Consider, for instance, the equation
\begin{equation}
	-\alpha \nabla^2 u = f \quad \text{in} \; \Omega 
	\label{eq:heat-equation}
\end{equation}
where $u = u(\bm{x})$ and $f = f(\bm{x})$ are both functions with respect to the vector of space variables $\bm{x} = (x_1, x_2, \dots, x_n)^T$ and $\Omega \supset \mathbb{R}^d$.
Equation~\eqref{eq:heat-equation} describes the temperature distribution inside a medium whose thermal conductivity is determined by the coefficient $\alpha$ and which contains a heat source $f$.
Since Equation~\eqref{eq:heat-equation} is only satisfied in the interior of the domain $\Omega$, we, additionally need to define a set of conditions at its boundaries.
These so-called \emph{boundary conditions} (BCs) can be usually classified in four different types:
\begin{description}
	\item[Dirichlet] $u(\bm{x}) = g(\bm{x})$
	\item[Neumann] $\frac{\partial}{\partial \vec{n}} u(\bm{x}) = 0$
	\item[Robin] $a u(\bm{x}) + b \frac{\partial}{\partial \vec{n}} u(\bm{x}) = g(\bm{x})$
	\item[Cauchy] $a u(\bm{x}) = g(\bm{x}), \; b \frac{\partial}{\partial \vec{n}} u(\bm{x}) = h(\bm{x})$
\end{description}
Here $\frac{\partial}{\partial \vec{n}} u(\bm{x})$ denotes the partial derivative of $u$ with respect to outwards-directed normal vector of the boundary.
Note that the difference between Robin and Cauchy BCs is that in the former case one condition is formulated as a weighted average of $u$ and its derivative in normal direction, while for the latter two conditions must be met individually.
While depending on the boundary conditions an analytical solution for Equation~\eqref{eq:heat-equation} might exist, for many PDEs such a solution has not been discovered or its computation is unfeasible.
A remedy is the application of so-called numerical methods, that are based on approximating the solution of a given PDE on a discrete set of grid points. 
\subsection{Grid Creation}
Before computing a numerical approximation of the solution of a given PDE, we need to define a set of discrete points within the domain $\Omega$ at which we aim to obtain the solution.
Usually, these points are defined on a grid or mesh with certain structure.
In general, a distinction is made between structured (or regular) and unstructured grids.
A structured grid is characterized by the uniform neighborhood of its grid points, which means that the number of neighbors is typically the same for each grid point.
In contrast, each point within an unstructured grid can have a varying number of neighbors.
Computations are usually easier to implement and more efficient on structured grids, due to their regularity.
However, structured grids are often difficult to create on complicated and irregular domains.
On the other hand, unstructured grids offer a higher degree of flexibility and are, therefore, also well-suited for the previously mentioned cases. %TODO include references
This thesis focuses on numerical methods that can be formulated on a hierarchy of structured grids. 
In the following, we, therefore, focus on this particular grid type.
One possibility to approximate a given PDE on a structured set of grid points is to compute the Taylor series expansion around each of them points, which leads to the so-called \emph{finite difference method} (FDM).
\subsection{Finite Difference Method}
For the sake of simplicity, we consider the one-dimensional function $u(x)$.
To obtain an approximation for the derivatives of $u$, we can compute its Taylor expansion in the neighborhood of $x$ with a step size $h$, which yields
\begin{equation}
	u(x + h) = u(x) - h \dv{x} u(x) + \mathcal{O}(h^2)
\end{equation}
Assuming $h$ is sufficiently small, the first-order approximation 
\begin{equation}
	\dv{x} u(x) \approx \frac{u(x + h) -  u(x)}{h}
\end{equation}
is obtained.
Furthermore, we can derive an approximation for the second-order partial derivative $\dv[2]{x}$ by considering
\begin{equation}
	u(x + h) = u(x) + h \dv{x} u(x) + \frac{h^2}{2} \dv[2]{x} u(x) + \mathcal{O}(h^3)
	\label{eq:taylor-forward}
\end{equation}
and 
\begin{equation}
	u(x - h) = u(x) - h \dv{x} u(x) + \frac{h^2}{2} \dv[2]{x} u(x) + \mathcal{O}(h^3).
	\label{eq:taylor-backward}
\end{equation}
Adding Equation~\eqref{eq:taylor-backward} to Equation~\eqref{eq:taylor-forward} then yields the second-order finite difference approximation
\begin{equation}
	 \dv[2]{x} u(x) \approx \frac{u(x + h) + u(x - h) - 2u(x)}{h^2}.
\end{equation}
Using the same technique similar approximation terms can be obtained for higher-dimensional functions and higher-order derivatives. %TODO Referenz einfügen
While finite differences offer a simple and straightforward way to approximate a given PDE on a set of structured grid points, in many cases the underlying physical requirements and complex geometries necessitate the use of semi-structured or even unstructured grids together more complicated approaches such as the finite volume (FVM) and finite element method (FEM) together with the use of semi- or unstructured grids.
For an in-depth introduction to these methods, the reader is referred to one of the following references:
%TODO insert references about FVM and FEM
\subsection{Model Problem}
To illustrate the previous two sections, we consider the following model problem, which represents a two-dimensional version of Poisson's equation:
\begin{equation}
	\begin{split}
		-\frac{\partial^2}{\partial x^2} u(x,y) - \frac{\partial^2}{\partial y^2} u(x,y) & = f(x, y) \quad \forall x, y \in (0, 1) \\
		u(0, y) = u(x, 0) = u(1, y) = u(x, 1) & = 0 \quad \forall x, y \in (0, 1)
	\end{split}
	\label{eq:2D-poisson-model}
\end{equation}
Note that this equation corresponds to the two-dimensional steady-state heat equation with constant $\alpha = 1$ on the unit square $\Omega = ( 0, 1 )^2$.
We then discretize Equation~\eqref{eq:2D-poisson-model} using finite differences and a uniform step size $h$, which yields
\begin{equation}
	\begin{split}
		\frac{1}{h^2} (4 u_{i,j} - u_{i-1, j} - u_{i+1, j} - u_{i, j-1} - u_{i, j+1}) & = f_{i, j} \quad \forall i, j \in \{1, 2, \dots n\} \\
		u_{0, j} = u_{i, 0} = u_{n+1, j} = u_{i, n+1} & = 0 \quad \forall i, j \in \{1, 2, \dots n\}
	\end{split} 
	\label{eq:2D-poisson-model-discrete}
\end{equation}
with $u_{i,j} = u(ih, jh)$, $f_{i,j} = f(ih, jh)$ and $n = 1/h - 1$.
By defining a unique ordering of the grid points $u_{i, j}$, Equation~\eqref{eq:2D-poisson-model-discrete} we can be transformed to a system of linear equation of type $A \bm{u} = \bm{f}$. 
For instance, setting $h = 0.25$, results in $n = 3$ and, hence, a total number of nine grid points.
%TODO draw grid
Assuming a natural ordering of grid points, we can represent the resulting system of linear equations as follows
\begin{equation}
\frac{1}{h^2} \underbrace{ \begin{pmatrix}
\begin{array}{ccc|ccc|ccc}~4&-1&~0&-1&~0&~0&~0&~0&~0\\-1&~4&-1&~0&-1&~0&~0&~0&~0\\~0&-1&~4&~0&~0&-1&~0&~0&~0\\\hline -1&~0&~0&~4&-1&~0&-1&~0&~0\\~0&-1&~0&-1&~4&-1&~0&-1&~0\\~0&~0&-1&~0&-1&~4&~0&~0&-1\\\hline ~0&~0&~0&-1&~0&~0&~4&-1&~0\\~0&~0&~0&~0&-1&~0&-1&~4&-1\\~0&~0&~0&~0&~0&-1&~0&-1&~4\end{array}
	\end{pmatrix}}_{\textstyle A}
\underbrace{
	\begin{pmatrix}
	u_{11} \\ u_{12} \\ u_{13} \\ u_{21} \\ u_{22} \\ u_{23} \\ u_{31} \\ u_{32} \\ u_{33}
\end{pmatrix}}_{\textstyle{\bm{u}}} = \bm f
\label{eq:2D-poisson-assembled-matrix}
\end{equation}
with a right-hand side $\bm f$, which can be further reduced with respect to the given boundary conditions that assume a constant value of zero at all boundary points:
\begin{equation}
\bm f = \begin{pmatrix}
		f_{11} + \frac{1}{h^2} (u_{10} + u_{01}) \\f_{12} + \frac{1}{h^2} u_{02} \\ f_{13} + \frac{1}{h^2} (u_{03} + u_{14})  \\ f_{21} + \frac{1}{h^2} u_{20} \\ f_{22} \\ f_{23} + \frac{1}{h^2} u_{24} \\ f_{31} + \frac{1}{h^2} (u_{30} + u_{41}) \\ f_{32} + \frac{1}{h^2} u_{42} \\ f_{33} + \frac{1}{h^2} (u_{34} + u_{43})
\end{pmatrix} = \begin{pmatrix}
f_{11} \\
f_{12} \\ 
f_{13} \\ 
f_{21} \\ 
f_{22} \\
f_{23} \\
f_{31} \\ 
f_{32} \\
f_{33}
\end{pmatrix}.
\label{eq:2D-poisson-assembled-rhs}
\end{equation}
Note that $A$ represents a sparse matrix, which means that only a minority of its entries are nonzero.
Moreover, since we already represent the solution of Equation~\ref{eq:2D-poisson-model-discrete} on a regular grid, storing the resulting linear system in a matrix-vector format is inefficient.
If we assume that each operation required for solving a system of linear equations can be formulated in form of a matrix-vector or vector-vector operation, it is unnecessary to store the corresponding matrix explicitly.
Instead, we only have to store the computational pattern that corresponds to a matrix-vector multiplication performed on a vector with the respective ordering of the grid points.
On regular grids, each of such pattern can be represented as a so-called stencil code.
\subsection{Stencil Codes}
\label{subsec:stencil-codes}
In general, a stencil code can be considered as a set of mappings from an n-dimensional index vector to a complex-valued number.
More formally we can define the general stencil $S^{(m,d)}$ as a finite set of $m$ tuples of the following form:

\begin{equation}
	\begin{split}
			& S^{(m, d)} = \{(\bm{a}_k, b_k) \}_{k=1}^m = \{(\bm{a}_1, b_1),  (\bm{a}_2, b_2), \dots (\bm{a}_m, b_m)\} 
	\\ & \forall \, i, j, k \in \{1, 2, \dots m \}: \,
	\bm{a}_k \in \mathbb{Z}^d \wedge \bm{a}_i \neq \bm{a}_j \; \text{with} \; i \neq j, \; b_k \in \mathbb{C}.
	\end{split}
\end{equation}
Here, the left entry $\bm{a}_k$ of each tuple $(\bm{a}_k, b_k)$ denotes the \emph{offset} from the index of the current grid point and the left entry $b_k$ the respective \emph{weight} or \emph{value}.
For one- and two-dimensional problems, it is often more convenient to use the alternative notation 
\begin{equation}
	S^{(m, 1)} = \begin{bmatrix}
		\cdots & s_{-1} & s_{0} & s_{1} & \cdots
	\end{bmatrix}
\end{equation}
and
\begin{equation}
	S^{(m, 2)} = \begin{bmatrix}
		& \vdots & \vdots & \vdots & \\
		\cdots & s_{1,-1} & s_{1,0} & s_{1,1} & \cdots \\
		\cdots & s_{0,-1} & s_{0,0} & s_{0,1} & \cdots \\
		\cdots & s_{-1,-1} & s_{-1,0} & s_{-1,1} & \cdots \\
		& \vdots & \vdots & \vdots &
	\end{bmatrix}
\end{equation}
where for all $s_{\bm{a}_i}$  with $i \in \{1, \dots, m\}$ there exists a tuple $(\bm{a}_i, b_i) \in S^{(m, d)}$ such that $b_i = s_{\bm{a}_i}$.
We can also extend this notation to three dimensions, by representing all components with the same offset as a separate two dimensional stencil, such that
\begin{equation}
	S^{(m,3)} = 
	\begin{bmatrix}
		\cdots & S_{z-1}^{(m_{z-1},2)} & S_{z}^{(m_{z},2)} & S_{z+1}^{(m_{z+1},2)} & \cdots 
	\end{bmatrix},
\label{eq:3D-stencil-matrix-notation}
\end{equation}
where $m = \sum_{i} m_{z+i}$.
%TODO specify range of i

Furthermore, we define the application of a stencil $S^{(m, d)}$ to a certain point $u_{\bm i}$ with $\bm i \in \mathcal G$, where $\mathcal G \supset \mathbb{N}^d$ is the set of grid indices, each considered as an n-dimensional vector:
\begin{equation}
	\begin{split}
		& S^{(m, d)} \bullet u_{\bm i} = \sum_{k=1}^m b_k u_{\bm i + \bm{a}_k} \quad 
		\text{with} \; \bm{i} \in \mathcal G \supset \mathbb{N}^d, \\ & (\bm{a}_i, b_i) \in S^{(m,d)} \; \forall \, i \in \{ 1, 2, \dots m \} 
	\end{split}
\label{eq:stencil-application}
\end{equation}
For $d = 3$, we can formulate the application of a three-dimensional stencil $S^{(m,3)}$, based on Equation~\eqref{eq:3D-stencil-matrix-notation}, as a dot product of the form
\begin{equation}
	S^{(m, 3)} \bullet u_{x,y,z} = 	
\begin{bmatrix}
	\cdots & S_{z-1}^{(m_{z-1},2)} & S_{z}^{(m_{z},2)} & S_{z+1}^{(m_{z+1},2)} & \cdots 
\end{bmatrix} \bullet
\begin{bmatrix}
	\vdots \\ u_{x,y,z-1} \\ u_{x,y,z} \\ u_{x,y,z+1} \\ \vdots 
\end{bmatrix},
\end{equation}
where each element-wise multiplication is interpreted as an application of the operator defined in Equation~\eqref{eq:stencil-application}.

As a two-dimensional example we consider the following five-point stencil:
\begin{equation}
	\begin{split}
		\Delta_h^{(5,2)} = & \bigg\{ \left( \left( 0,0 \right), \frac{4}{h^2}\right), \left(\left(1,0\right), \frac{-1}{h^2}\right), \left(\left(-1,0\right), \frac{-1}{h^2}\right), \\ & \left(\left(0,1\right), \frac{-1}{h^2}\right), \left(\left(0,-1\right), \frac{-1}{h^2}\right) \bigg\}
	\end{split}
	\label{eq:five-point-stencil}
\end{equation}
Applying $\Delta_{h}^{(5,2)}$ to a given point $u_{i,j}$ on a two-dimensional regular grid yields 
\begin{equation}
	\Delta_h^{(5,2)} \bullet u_{i,j} = \frac{1}{h^2} \left(4 u_{i+0,j+0}  - u_{i-1,j+0} - u_{i+1,j+0} - u_{i+0,j-1} - u_{i+0,j-1}\right),
\end{equation}
which corresponds precisely to the left part of Equation~\eqref{eq:2D-poisson-model-discrete}.
From this example it becomes evident that applying a given stencil to set of points $u_{\bm x}$ defined on a regular grid can be always considered as a sparse matrix-vector product, where the matrix is obtained by sorting the grid points according to a well-defined order, which has been already demonstrated in Equation~\eqref{eq:2D-poisson-assembled-matrix}.
Therefore, in case it is possible to define each computational step for solving a given discretized PDE on a regular grid by means of a stencil application, we only need to obtain the respective stencil entries instead of assembling or even storing a complete matrix.
Furthermore, many operators defined for matrices can be easily transferred into the domain of stencil codes.
As a first elementary operation, we define the multiplication of a stencil with a scalar, which is simply achieved by multiplying the weight of each entry with the respective value:
% Scalar multiplication
\begin{equation}
	\alpha S^{(m, d)} = \{(\bm{a}_k, \alpha b_k) \}_{k=1}^m
\end{equation}

Next, we can define certain binary operations such as the addition, subtraction and multiplication of two stencils.
Equation~\eqref{eq:stencil-addsub} contains a recursive definition of stencil addition and subtraction.
Here, we simply add the weights of each entry with the same offset contained in both $S_1$ and $S_2$.
In case an offset is not contained in both stencils the respective tuple is included unmodified.  
% Stencil addition and subtraction
\begin{equation}
	S_1 \pm S_2 = 
	\begin{cases}
		\{(\bm{a}, b\pm c ) \} \cup (\tilde{S}_1 \pm \tilde{S}_2) & \text{if} \; S_1 = 	\{(\bm{a}, b ) \} \cup \tilde{S}_1 \\
		& \text{and} \; S_2 = \{(\bm{a}, c ) \} \cup \tilde{S}_2 \\
		
		\{(\bm{a}, b ) \} \cup (\tilde{S}_1 \pm S_2) & \text{if} \; S_1 = 	\{(\bm{a}, b ) \} \cup \tilde{S}_1 \\
		& \text{and} \; \{(\bm{a}, c ) \} \notin S_2 \;
		\forall c \in \mathbb{C}
		\\
% TODO case probably unnecessary
%		\{(\bm{a}, c ) \} \cup (S_1 \pm \tilde{S}_2) & \text{if} \; S_2 = 	\{(\bm{a}, c ) \} \cup \tilde{S}_2 \wedge
%		\{(\bm{a}, b ) \} \notin S_1 \; \forall b \in \mathbb{C}
%		\\
		S_1 & \text{if} \; S_2 = \emptyset
		\\
		S_2 & \text{else} 
		\\
	\end{cases}
\label{eq:stencil-addsub}
\end{equation}
Based on Equation~\eqref{eq:stencil-addsub} we can then also provide a definition of the multiplication of two stencils $S_1$ and $S_2$, which can be seen as a recursive version of the iterative computation scheme described in~\cite{rittich2018extending}:
% Stencil multiplication
\begin{equation}
	S_1 \cdot S_2 = 
	\begin{cases}
		\{(\bm{a} + \bm{\hat{a}}, b \cdot \hat{b} ) \} +  \{(\bm{a}, b ) \} \cdot \tilde{S}_2 + \tilde{S}_1 \cdot S_2 & \text{if} \; S_1 = \{(\bm{a}, b ) \} \cup \tilde{S}_1 \\
		&
		\text{and} \; S_2 = \{(\bm{\hat{a}}, \hat{b} ) \} \cup \tilde{S}_2 \\
		\emptyset & \text{if} \; S_1 = \emptyset \vee S_2 = \emptyset
	\end{cases}
\label{eq:stencil-mult}
\end{equation}
Here, to compute the stencil product $S_1 \cdot S_2$, for each pair of entries $\{(\bm{a}, b ) \} \in S_1$ and $\{(\bm{\hat{a}}, \hat{b} ) \} \in S_2$ the tuple $\{(\bm{a} + \bm{\hat{a}}, b \cdot \hat{b} ) \}$ needs to be formed.
To define this computation recursively, we first pick random entries $\{(\bm{a}, b ) \} \in S_1$ and $\{(\bm{\hat{a}}, \hat{b} ) \} \in S_2$ and combine them as described above.
We then multiply the entry $\{(\bm{a}, b ) \}$ chosen from $S_1$ with the remainder stencil $\tilde{S}_2 = S_2 \setminus \{(\bm{\hat{a}}, \hat{b} ) \}$, which means that we now obtain the combination of this entry with all remaining ones from $S_2$.
Finally, the process is continued recursively by computing the product of the remainder $\tilde{S}_1 = S_1 \setminus \{(\bm{a}, b ) \}$, with the original stencil $S_2$.
Since it is possible that the combination of different pairs of stencil offsets leads to the same result, we, additionally, need to sum up the corresponding weights and combine them in a single tuple.
For this purpose, we employ stencil addition, as defined in Equation~\eqref{eq:stencil-addsub}, to sum up all tuples with matching offset obtained within each recursive subcomputation.
%TODO introduce array stencil notation for 1D and 2D
\subsection{Systems of Partial Differential Equations}
While so far we have only considered problems formulated in form of a single partial differential equation, many phenomena can only be modeled in form of a system of PDEs.
One of the most simple systems of PDEs is the so-called biharmonic equation
\begin{equation}
	\begin{split}
		\nabla^2 u & = v  \\
		\nabla^2 v & = f.
	\end{split}
\label{eq:biharmonic-system}
\end{equation}
As a remark, note that this system is mathematically equivalent to the scalar equation 
\begin{equation}
	\nabla^4 u = f.
\end{equation}
We can reformulate Equation~\eqref{eq:biharmonic-system} to obtain
\begin{equation}
	\underbrace{
	\begin{pmatrix}
		\nabla^2 & -1 \\
		0 & \nabla^2
	\end{pmatrix}}_{A}
\underbrace{ 
	\begin{pmatrix}
		u \\ v
	\end{pmatrix}
}_{\bm{u}}
=
\underbrace{
\begin{pmatrix}
	0 \\ f
\end{pmatrix}
}_{\bm{f}}.
\label{eq:biharmonic-system-matrix-formulation}
\end{equation}
While in contrast to our previous examples Equation~\eqref{eq:biharmonic-system-matrix-formulation} includes partial derivatives of multiple variables, we can employ the same techniques to discretize each individual operator and obtain a corresponding system of linear equations. 
Consider the two-dimensional biharmonic system
\begin{equation}
	\begin{split}
		\frac{\partial^2}{\partial x^2} u(x,y) + \frac{\partial^2}{\partial y^2} u(x,y) - v(x, y) & = 0 \\
		\frac{\partial^2}{\partial x^2} v(x,y) + \frac{\partial^2}{\partial y^2} v(x,y) & = f(x, y) \\ \forall x, y \in (0, 1)^2
	\end{split}
	\label{eq:2D-biharmonic-system}
\end{equation}
Discretizing Equation~\eqref{eq:2D-biharmonic-system} using finite differences with a uniform step size $h$ and rewriting the resulting equations in stencil form yields
\begin{equation}
		\begin{pmatrix}
			\Delta_h^{(5,2)} & -1 \\
			0 & \Delta_h^{(5,2)}
	\end{pmatrix}
		\begin{pmatrix}
			u_{i,j} \\ v_{i,j}
		\end{pmatrix}
	=
		\begin{pmatrix}
			0 \\ f_{i,j}
		\end{pmatrix} \\
\forall i,j \in \{1, 2, \dots n\},
	\label{eq:2D-biharmonic-system-stencil}
\end{equation}
where $\Delta_h^{(5,2)}$ is the two-dimensional five-point stencil defined in Equation~\eqref{eq:five-point-stencil}.
Equation~\eqref{eq:2D-biharmonic-system-stencil} then represents a system of two linear equations which needs to be solved at every pair of grid points $u_{i,j}$ and $v_{i,j}$.
If we consider Equation~\eqref{eq:2D-biharmonic-system-stencil} at all grid points at once, a system of linear equations is obtained whose solution corresponds to the discrete solution of Equation~\eqref{eq:2D-biharmonic-system}.
Before we conclude this section, it is important to mention that we have not yet discussed how each variable, in the given case $u_{i,j}$ and $v_{i,j}$, is placed on the grid.
While it is tempting to always place each variable at the same position within a grid, in many applications this leads to undesirable numerical properties.%TODO insert reference
As a consequence, often more complex grid placing strategies are used in practice, for instance so-called staggered grids, which we, for the sake of brevity, do not discuss here.
For a detailed treatment of these techniques, the reader is referred to %TODO insert suitable reference
.
Also note that in order to solve Equation~\eqref{eq:2D-biharmonic-system}, or any other system of PDEs, we need to define suitable conditions at the boundaries of each domain on which a given variable, contained in one of the equations, is defined.
These boundary conditions can be of the same type as described above.
Again, for a more detailed treatment of boundary conditions for systems of PDEs, the reader is referred to %TODO insert suitable reference
.
 


