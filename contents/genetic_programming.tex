\section{Genetic Programming}
\emph{Genetic Programming} (GP) is a class of search algorithms for finding optimal programs based on the principle of natural evolution first proposed by John Koza~\cite{koza1994genetic}.
In general, a program $p$ can be considered as a mapping from the set of inputs $\mathcal{I}$ to the corresponding set of outputs $\mathcal{O}$
\begin{equation}
	p : \mathcal{I} \to \mathcal{O}.
\end{equation}
However, since not all input-output pairs are known in advance, this mapping is usually given in form of a finite set of training cases.
The goal of GP is then to find a program that correctly computes the correct output for each input contained in the training set.
Since there is often not a unique program that satisfies this condition, usually a number of additional constraints are applied to assess the quality of each correct program.
For instance, in practice one is often interested in finding the shortest or fastest program that is able to pass all training cases.
Based on a program's effectiveness in solving all training cases under the given constraints, GP then assigns a \emph{fitness} value to each program, which is then treated as an \emph{individual} within a \emph{population} of programs.
In evolutionary computation, the population is a set of individuals currently considered within the search and, therefore, spans a subspace within the space of solutions for the given search problem.
Each subsequent step of the search is then performed by generating a new population based on the previous one through \emph{mutation} or recombination (often called \emph{crossover}) of the individuals contained in the current one.
Usually, the candidates for mutation and crossover are sampled from the current population based on the fitness value computed for each individual.
If we repeat this procedure for a number of $n$ steps, we arrive at a final population $P_n$, which can then additionally be evaluated on a test set, to obtain the best overall program.
The resulting search method is summarized in Algorithm~\ref{alg:genetic-programming}. 
\begin{algorithm}[t]
	\caption{Genetic Programming}
	\label{alg:genetic-programming}
	\begin{algorithmic} % The number tells where the line numbering should start
		\State \textbf{Randomly generate} an initial population $P_0$ of programs
		\State \textbf{Evaluate} $P_0$ on the \textbf{training set} 
		\For{$i := 1, \dots, n$}
		\State \textbf{Select} a subset of individuals $M_i \supset P_{i-1}$ based on their \textbf{fitness}
		\State \textbf{Generate} new programs $C_i$ based on $M_i$ using \textbf{mutation} and \textbf{crossover}
		\State \textbf{Evaluate} $C_i$ on the \textbf{training set} 
		\State \textbf{Select} $P_{i}$ from $C_i \cup P_{i-1}$
		\EndFor
		\State \textbf{Evaluate} the final population $P_{n}$ on a \textbf{test set}  to obtain the best overall program
	\end{algorithmic}
\end{algorithm}
While Algorithm~\ref{alg:genetic-programming} gives an overview of the general structure of a GP method, we have not yet considered how the individual operations such as the generation of an initial population and the creation of new individuals through mutation and crossover can be performed.
Since all these operations are based on manipulating the internal structure of a given program, the first step in the implementation of a GP method choosing a suitable program representation.
Note that this representation does not necessarily need to be equal to the target language in which the actual program is supposed to be implemented but rather needs to define a unique mapping that enables its automatic generation.
This process is usually called \emph{genotype} to \emph{phenotype} mapping, where genotype refers to the internal representation used within GP while phenotype then represents the actual program implemented on the target machine.
One of the most widely used genotype representations is tree-based GP, where each program is internally represented as a tree of expressions, which was also initially proposed by John Koza~\cite{koza1994genetic}.
While within the last decades numerous other GP variants have been proposed~\cite{poli2008field}, such as grammatical evolution~\cite{o2001grammatical}, linear GP~\cite{brameier2007linear} and cartesian GP~\cite{miller2008cartesian}, we focus on tree-based GP as it forms the basis for our implementation, which will be presented in later chapters.
